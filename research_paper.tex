\documentclass[11pt,a4paper,twocolumn]{article}

% ================================
% PACKAGES
% ================================
\usepackage[utf8]{inputenc}
\usepackage[T1]{fontenc}
\usepackage[romanian]{babel}
\usepackage{amsmath, amssymb, amsthm}
\usepackage{graphicx}
\usepackage{subcaption}
\usepackage{hyperref}
\usepackage{geometry}
\usepackage{abstract}
\usepackage{biblatex}
\usepackage{csquotes}
\usepackage{xcolor}
\usepackage{booktabs}
\usepackage{algorithm}
\usepackage{algpseudocode}
\usepackage{multirow}
\usepackage{todonotes}
\usepackage{listings}

% ================================
% PAGE SETUP
% ================================
\geometry{
    left=1.5cm,
    right=1.5cm,
    top=2.5cm,
    bottom=2.5cm
}

% ================================
% BIBLIOGRAPHY
% ================================
\addbibresource{references.bib}

% ================================
% HYPERLINK SETTINGS
% ================================
\hypersetup{
    colorlinks=true,
    linkcolor=blue,
    urlcolor=blue,
    citecolor=blue,
    filecolor=blue
}

% ================================
% CODE LISTING SETTINGS
% ================================
\lstset{
    basicstyle=\ttfamily\small,
    keywordstyle=\color{blue},
    commentstyle=\color{gray},
    stringstyle=\color{red},
    numbers=left,
    numberstyle=\tiny,
    stepnumber=1,
    numbersep=5pt,
    frame=single,
    breaklines=true,
    showstringspaces=false
}

% ================================
% METADATA
% ================================
\title{Arhitectura de Securitate a Platformei Android: Analiza Mecanismelor de Protecție}

\author{%
    Budiul Cristian-Carol\\[0.1cm]
    Titianu Maria\\[0.3cm]
}

\date{\today}

% ================================
% CUSTOM COMMANDS
% ================================
\newcommand{\dataset}[1]{\textbf{#1}}

% Theorem environments
\theoremstyle{definition}
\newtheorem{definition}{Definiție}[section]
\newtheorem{example}{Exemplu}[section]

\theoremstyle{plain}
\newtheorem{theorem}{Teoremă}[section]
\newtheorem{lemma}{Lemă}[section]
\newtheorem{proposition}{Propoziție}[section]
\newtheorem{corollary}{Corolar}[section]

\theoremstyle{remark}
\newtheorem{remark}{Observație}[section]

% Algorithm naming
\algrenewcommand\algorithmicfunction{\textbf{funcție}}
\algrenewcommand\algorithmicprocedure{\textbf{procedură}}

% ================================
% DOCUMENT
% ================================
\begin{document}

\maketitle

% ================================
% ABSTRACT (ROMANIAN)
% ================================
\begin{abstract}

\end{abstract}

\textbf{Cuvinte cheie:}

% ================================
% 2. INTRODUCERE
% ================================
\section{Introducere}

Lorem ipsum dolor sit amet, consectetur adipiscing elit. Sed do eiusmod tempor incididunt ut labore et dolore magna aliqua. Ut enim ad minim veniam, quis nostrud exercitation ullamco laboris nisi ut aliquip ex ea commodo consequat. Duis aute irure dolor in reprehenderit in voluptate velit esse cillum dolore eu fugiat nulla pariatur. Excepteur sint occaecat cupidatat non proident, sunt in culpa qui officia deserunt mollit anim id est laborum. Sed ut perspiciatis unde omnis iste natus error sit voluptatem accusantium doloremque laudantium, totam rem aperiam, eaque ipsa quae ab illo inventore veritatis et quasi architecto beatae vitae dicta sunt explicabo. Nemo enim ipsam voluptatem quia voluptas sit aspernatur aut odit aut fugit, sed quia consequuntur magni dolores eos qui ratione voluptatem sequi nesciunt. Neque porro quisquam est, qui dolorem ipsum quia dolor sit amet, consectetur, adipisci velit, sed quia non numquam eius modi tempora incidunt ut labore et dolore magnam aliquam quaerat voluptatem.

At vero eos et accusamus et iusto odio dignissimos ducimus qui blanditiis praesentium voluptatum deleniti atque corrupti quos dolores et quas molestias excepturi sint occaecati cupiditate non provident, similique sunt in culpa qui officia deserunt mollitia animi, id est laborum et dolorum fuga. Et harum quidem rerum facilis est et expedita distinctio. Nam libero tempore, cum soluta nobis est eligendi optio cumque nihil impedit quo minus id quod maxime placeat facere possimus, omnis voluptas assumenda est, omnis dolor repellendus. Temporibus autem quibusdam et aut officiis debitis aut rerum necessitatibus saepe eveniet ut et voluptates repudiandae sint et molestiae non recusandae. Itaque earum rerum hic tenetur a sapiente delectus, ut aut reiciendis voluptatibus maiores alias consequatur aut perferendis doloribus asperiores repellat.

Lorem ipsum dolor sit amet, consectetur adipiscing elit, sed do eiusmod tempor incididunt ut labore et dolore magna aliqua. Ut enim ad minim veniam, quis nostrud exercitation ullamco laboris nisi ut aliquip ex ea commodo consequat. Duis aute irure dolor in reprehenderit in voluptate velit esse cillum dolore eu fugiat nulla pariatur. Excepteur sint occaecat cupidatat non proident, sunt in culpa qui officia deserunt mollit anim id est laborum. Sed ut perspiciatis unde omnis iste natus error sit voluptatem accusantium doloremque laudantium, totam rem aperiam, eaque ipsa quae ab illo inventore veritatis et quasi architecto beatae vitae dicta sunt explicabo. Nemo enim ipsam voluptatem quia voluptas sit aspernatur aut odit aut fugit, sed quia consequuntur magni dolores eos qui ratione voluptatem sequi nesciunt.

Neque porro quisquam est, qui dolorem ipsum quia dolor sit amet, consectetur, adipisci velit, sed quia non numquam eius modi tempora incidunt ut labore et dolore magnam aliquam quaerat voluptatem. Ut enim ad minima veniam, quis nostrum exercitationem ullam corporis suscipit laboriosam, nisi ut aliquid ex ea commodi consequatur? Quis autem vel eum iure reprehenderit qui in ea voluptate velit esse quam nihil molestiae consequatur, vel illum qui dolorem eum fugiat quo voluptas nulla pariatur. At vero eos et accusamus et iusto odio dignissimos ducimus qui blanditiis praesentium voluptatum deleniti atque corrupti quos dolores et quas molestias excepturi sint occaecati cupiditate non provident, similique sunt in culpa qui officia deserunt mollitia animi, id est laborum et dolorum fuga.

\subsection{Contextul General: Creșterea Dependenței de Dispozitive Mobile}

\subsection{Obiectivele Lucrării și Motivația Alegerii Temei}

\paragraph{Obiettive principale:}

\paragraph{Motivația alegerii temei:}

% ================================
% 3. PREZENTAREA PLATFORMEI ANDROID
% ================================
\section{Prezentarea Platformei Android}

\subsection{Arhitectura Sistemului}

\subsubsection{Kernel Linux}

\subsubsection{Biblioteci Native}

\subsubsection{Mediul de Execuție (Dalvik/ART)}

\subsubsection{Framework-ul Aplicațiilor}

\subsubsection{Stratul de Aplicații}

\subsection{Mecanisme de Izolare}

\subsubsection{Procese Separate}

\subsubsection{UID Distinct pentru Fiecare Aplicație}

\subsection{Sistemul de Permisiuni și Semnarea Aplicațiilor}

\subsubsection{Permisiuni Android}

\subsubsection{Semnarea Aplicațiilor}

% ================================
% 4. MODELUL DE SECURITATE AL PLATFORMEI ANDROID
% ================================
\section{Modelul de Securitate al Platformei Android}

\subsection{Mecanisme de Bază}

\subsubsection{Sandboxing}

\subsubsection{Permisiuni}

\subsubsection{Izolare de Procese și Fișiere}

\subsection{Extensii Moderne}

\subsubsection{SELinux / SEAndroid}

\subsubsection{Verified Boot}

\subsubsection{Criptarea Datelor (FDE/FBE)}

\subsubsection{Keystore \& TEE}

% ================================
% 5. VULNERABILITĂȚI ȘI AMENINȚĂRI SPECIFICE
% ================================
\section{Vulnerabilități și Amenințări Specifice}

\subsection{Vulnerabilități ale Sistemului și Aplicațiilor}

\subsubsection{Exploatarea Privilegiilor}

\subsubsection{Vulnerabilități Kernel/Driver}

\subsection{Malware și Aplicații Malițioase}

\subsubsection{DroidDream (2011)}

\subsubsection{GingerMaster (2011)}

\subsubsection{Masque Attack (2014)}

\subsection{Riscuri legate de Root/Jailbreak și ROM-uri Terțe}

\subsection{Pierderea Confidențialității Datelor Personale}

\subsubsection{Tracking}

\subsubsection{Acces Neautorizat}

% ================================
% 6. STUDII DE CAZ - VULNERABILITĂȚI REPREZENTATIVE
% ================================
\section{Studii de Caz – Vulnerabilități Reprezentative}

\subsection{Stagefright (2015) - Atac Media}

\subsection{QuadRooter (2016) - Vulnerabilități în Drivers}

% ================================
% 7. MĂSURI DE PROtecție ȘI SOLUȚII PROPUSE
% ================================
\section{Măsuri de Protecție și Soluții Propuse}

\subsection{La Nivel de Utilizator: Actualizări, Permisiuni, Igienă Digitală}

\subsubsection{Actualizări}

\subsubsection{Permisiuni}

\subsubsection{Igienă Digitală}

\subsection{La Nivel de Dezvoltator: Principiul Minimului Privilegiu, Criptare, Protecție IPC}

\subsubsection{Principiul Minimului Privilegiu}

\subsubsection{Criptarea Datelor}

\subsubsection{Protecție IPC}

\subsection{La Nivel de Sistem: Întărirea Kernel (SELinux), Verificarea Aplicațiilor (Google Play Protect)}

\subsubsection{Întărirea Kernel}

\subsubsection{Google Play Protect}

% ================================
% 8. CONCLUZII
% ================================
\section{Concluzii}

\subsection{Sinteză a Observațiilor}

\subsection{Previziuni: Securitate Bazată pe AI, Sandboxing Hardware, Izolarea Datelor prin TEE, RKP}

\subsection{Final Thoughts}

% ================================
% 9. BIBLIOGRAFIE
% ================================
\section{Bibliografie}

% ================================
% 10. CONTRIBUȚIA AUTORILOR
% ================================
\section*{Contribuția Autorilor}

% ================================
% REFERENCES
% ================================
\printbibliography

% ================================
% APPENDICES
% ================================
\appendix

\section{Detalii Tehnice Suplimentare}

\subsection{Exemplu de Politică SELinux}

\section{Glosar de Termeni}

\end{document}
