\documentclass[11pt,a4paper]{article}

% Packages
\usepackage[utf8]{inputenc}
\usepackage[Times New Roman]{fontenc}
\usepackage{amsmath, amssymb, amsthm}
\usepackage{graphicx}
\usepackage{hyperref}
\usepackage{geometry}
\usepackage{abstract}
\usepackage{biblatex}
\usepackage{csquotes}

% Page margins
\geometry{
    left=2.5cm,
    right=2.5cm,
    top=3cm,
    bottom=3cm
}

% Bibliography
\addbibresource{references.bib}

% Metadata
\title{Title of Your Research Paper}
\author{Your Name\thanks{Institution Name}}
\date{\today}

% Custom commands
\theoremstyle{definition}
\newtheorem{definition}{Definition}[section]
\newtheorem{theorem}{Theorem}[section]
\newtheorem{lemma}{Lemma}[section]
\newtheorem{proof}{Proof}[section]

\begin{document}

% Title page
\maketitle

% Abstract
\begin{abstract}
This is the abstract of your research paper. It should provide a concise summary of your work, including the problem addressed, methodology, key findings, and conclusions. Aim for 150-250 words that capture the essence of your contribution.
\paragraph{Keywords:} keyword1, keyword2, keyword3, keyword4
\end{abstract}

% Main content starts
\section{Introduction}

The introduction should set the context for your research, motivate the problem, and outline your contributions. Start with the broader context and gradually narrow down to your specific research question.

\begin{itemize}
    \item State the research problem or question
    \item Explain the significance and motivation
    \item Review relevant prior work
    \item Outline your contributions
    \item Provide an overview of the paper structure
\end{itemize}

\subsection{Background}

Provide background information necessary for understanding your work. Explain key concepts, terminology, and the state of the field.

\subsection{Problem Statement}

Clearly articulate the problem you are addressing and why it matters.

\subsection{Contributions}

List your main contributions:
\begin{enumerate}
    \item First contribution
    \item Second contribution
    \item Third contribution
\end{enumerate}

% Literature Review (optional section)
\section{Related Work}

Review relevant literature and position your work within the existing body of knowledge. Compare and contrast your approach with previous work.

\subsection{Previous Approaches}

Discuss previous approaches to similar problems.

\subsection{Differences and Novelty}

Highlight how your approach differs from existing methods and what makes it novel.

\section{Methodology}

Describe your research methodology in detail. This section should be clear enough for others to replicate your work.

\subsection{Research Design}

Explain the overall design of your study or system.

\subsection{Data Collection and Processing}

If applicable, describe how data was collected, preprocessed, and analyzed.

\subsection{Implementation}

Provide technical details about implementation, tools used, and experimental setup.

\section{Results}

Present your findings clearly and systematically. Use tables, figures, and text to convey your results.

\subsection{Quantitative Results}

Here's an example of mathematical notation:
\begin{equation}
    E = mc^2
    \label{eq:einstein}
\end{equation}

And here's an in-line equation: $f(x) = \int_{-\infty}^{\infty} e^{-x^2} dx$

\subsection{Qualitative Analysis}

Discuss qualitative observations and their implications.

% Example table
\begin{table}[h]
    \centering
    \begin{tabular}{|l|c|c|}
        \hline
        \textbf{Metric} & \textbf{Method A} & \textbf{Method B} \\
        \hline
        Accuracy & 85\% & 92\% \\
        Precision & 0.87 & 0.91 \\
        Recall & 0.83 & 0.93 \\
        \hline
    \end{tabular}
    \caption{Comparison of different methods}
    \label{tab:comparison}
\end{table}

\subsection{Visualizations}

Include figures as needed:
% \begin{figure}[h]
%     \centering
%     \includegraphics[width=0.8\textwidth]{figure.png}
%     \caption{Your figure caption}
%     \label{fig:example}
% \end{figure}

\section{Discussion}

Interpret your results, discuss their implications, and address limitations.

\subsection{Interpretation of Results}

What do your results mean? How do they relate to your research questions?

\subsection{Limitations}

Acknowledge the limitations of your study:
\begin{itemize}
    \item What constraints exist in your methodology?
    \item What assumptions were made?
    \item What could not be addressed in this work?
\end{itemize}

\subsection{Future Work}

Suggest directions for future research:
\begin{itemize}
    \item Potential improvements
    \item Unanswered questions
    \item Suggested experiments or extensions
\end{itemize}

\section{Conclusion}

Summarize your main findings and contributions:
\begin{itemize}
    \item Recap the problem addressed
    \item Highlight key results
    \item Emphasize the significance of your contributions
    \item Suggest practical implications or applications
\end{itemize}

Your conclusion should tie everything together and leave readers with a clear understanding of what you've accomplished and why it matters.

% Acknowledgments (optional)
\section*{Acknowledgments}
We acknowledge [funding sources, collaborators, institutions, etc.].

% Bibliography
\printbibliography

% Appendices (optional)
\appendix
\section{Appendix A: Additional Details}
Additional materials, extended proofs, or supplementary information can go here.

\section{Appendix B: Code or Data}
Code listings, data tables, or other supplementary materials.

\end{document}
